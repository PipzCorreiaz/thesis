%!TEX TS-program = pdflatex
%!TEX encoding = UTF-8 Unicode

\usepackage[utf8]{inputenc} % set input encoding (not needed with XeLaTeX)

\usepackage[sort&compress,round]{natbib}
%\usepackage[square,numbers]{natbib}
\bibliographystyle{plainnat}
%\bibliographystyle{unsrtnat}

%for coloured hyperlinks
\usepackage{hyperref}
\hypersetup{ a4paper=true,
             colorlinks=false,
             citecolor=red,
             breaklinks=true,
             %bookmarks=true,
             bookmarksnumbered=true,
             bookmarksopen=true,
             pdftitle={Computational prediction of microRNA targets in plant genomes},                 % EDIT THIS LINE
             pdfauthor={Manuel Borges Dias dos Reis},                    % EDIT THIS LINE
             pdfsubject={Computational prediction of microRNA targets in plant genomes},                  % EDIT THIS LINE
             pdfcreator={Manuel Borges Dias dos Reis}, % EDIT THIS LINE
             pdfkeywords={miRNA, plants}                             % EDIT THIS LINE
}

%%% PAGE DIMENSIONS
\usepackage{geometry}
\usepackage{pdflscape}
\usepackage[T1]{fontenc}

\usepackage{graphicx} % support the \includegraphics command and options
\usepackage{caption}
\usepackage{subcaption}
\usepackage{wrapfig} % permite meter figuras no meio do texto

%% PACKAGE algorithmic, algorithm
%% ------------------------------
%% These packages are required if you need to describe an algorithm.
\usepackage{algorithmicx}
\usepackage{algorithm}
\usepackage[noend]{algpseudocode}
\usepackage{xspace}
\usepackage{setspace}
%\usepackage[chapter]{algorithm}

%\usepackage{citep}
\usepackage{float}
%\floatstyle{boxed}
%\restylefloat{figure}   % To re-style the 'figure' float
%\restylefloat{}    % To re-style the 'le' float.

%\usepackage[parfill]{parskip} % Activate to begin paragraphs with an empty line rather than an indent

\usepackage{pdfpages}

%%% PACKAGES
\renewcommand{\arraystretch}{1.2}
\usepackage{booktabs} % for much better looking tables
\usepackage{array} % for better arrays (eg matrices) in maths
\usepackage{paralist} % very flexible & customisable lists (eg. enumerate/itemize, etc.)
\usepackage{verbatim} % adds environment for commenting out blocks of text & for better verbatim
\usepackage{listings}
\usepackage{color}
 
\definecolor{codegreen}{rgb}{0,0.6,0}
\definecolor{codegray}{rgb}{0.5,0.5,0.5}
\definecolor{codepurple}{rgb}{0.58,0,0.82}
\definecolor{backcolour}{rgb}{0.95,0.95,0.92}
 
\lstdefinestyle{code}{
    backgroundcolor=\color{backcolour},   
    commentstyle=\color{codegreen},
    keywordstyle=\color{magenta},
    numberstyle=\tiny\color{codegray},
    stringstyle=\color{codepurple},
    basicstyle=\footnotesize,
    breakatwhitespace=false,         
    breaklines=true,                 
    captionpos=b,                    
    keepspaces=true,                 
    numbers=left,                    
    numbersep=5pt,                  
    showspaces=false,                
    showstringspaces=false,
    showtabs=false,                  
    tabsize=2
}
 
\lstset{style=code}

%\usepackage{subfig} % make it possible to include more than one captioned figure/table in a single float
\usepackage[table]{xcolor}% http://ctan.org/pkg/xcolor
\usepackage[flushleft,para]{threeparttable}
\usepackage{tabularx}
\usepackage{tablefootnote}
\usepackage{changepage}
\usepackage{array}

\newcolumntype{L}[1]{>{\raggedright\let\newline\\\arraybackslash\hspace{0pt}}m{#1}}
\newcolumntype{C}[1]{>{\centering\let\newline\\\arraybackslash\hspace{0pt}}m{#1}}
\newcolumntype{R}[1]{>{\raggedleft\let\newline\\\arraybackslash\hspace{0pt}}m{#1}}


\newcommand{\HRule}{\rule{\linewidth}{0.5mm}}

% AMS packages
\usepackage{amsmath}
\usepackage{amsfonts}
\usepackage{amssymb}


%%%%%%%%%%%%%%%%%%%%%%%%%%%%%%%%%%%%%%
%%%%%%%%%%%%%%% FILIPA %%%%%%%%%%%%%%%
%%%%%%%%%%%%%%%%%%%%%%%%%%%%%%%%%%%%%%
\usepackage[printonlyused]{acronym}
\usepackage{graphicx}
\usepackage{pifont}
\usepackage{multirow}
\usepackage{array}
\newcolumntype{L}[1]{>{\raggedright\let\newline\\\arraybackslash\hspace{0pt}}m{#1}}
\newcolumntype{C}[1]{>{\centering\let\newline\\\arraybackslash\hspace{0pt}}m{#1}}
\newcolumntype{R}[1]{>{\raggedleft\let\newline\\\arraybackslash\hspace{0pt}}m{#1}}
\newcommand*{\Comb}[2]{{}^{#1}C_{#2}}
\normalem
%%%%%%%%%%%%%%%%%%%%%%%%%%%%%%%%%%%%%%
%%%%%%%%%%%%%%%%%%%%%%%%%%%%%%%%%%%%%%
%%%%%%%%%%%%%%%%%%%%%%%%%%%%%%%%%%%%%%

\usepackage{fixfoot,footmisc}

\usepackage{enumitem}

\newcommand*\BitAnd{\mathrel{\&}}
\newcommand*\BitOr{\mathrel{|}}

\algnewcommand\algorithmicinitialize{\textbf{Initialize:}}
\algnewcommand\Initialize{\item[\algorithmicinitialize]}

\algnewcommand\algorithmicbasis{\textbf{Basis:}}
\algnewcommand\Basis{\item[\algorithmicbasis]}

\algnewcommand{\Print}{\textbf{print}\xspace}
\algnewcommand{\Increment}{\textbf{Increment}\xspace}
\algnewcommand{\By}{\textbf{by}\xspace}
\algnewcommand{\BinAnd}{\textbf{and}\xspace}
\algnewcommand{\BinOr}{\textbf{or}\xspace}
\algnewcommand{\BitOr}{\mathrel{|}}

\hyphenation{microRNA} % allows defining the hyphenation of a particular work
\hyphenation{microRNAs}
\hyphenation{pre-miRNA}
\hyphenation{miRNA}
\hyphenation{miRNAs}

\hoffset 0in
\voffset 0in
\oddsidemargin 0.71cm
\evensidemargin 0.04cm
\marginparsep 0in
\topmargin -0.25cm
\textwidth 15cm
\textheight 23.5cm

\usepackage{fancyhdr}
\pagestyle{fancy}
\renewcommand{\chaptermark}[1]{\markboth{\thechapter.\ #1}{}}
\renewcommand{\sectionmark}[1]{\markright{\thesection\ #1}}
\fancyhf{} \fancyhead[LE]{\bfseries\nouppercase{\leftmark}}
\fancyhead[RO]{\bfseries\nouppercase{\rightmark}}
\fancyfoot[LE,RO]{\bfseries\thepage}
\renewcommand{\headrulewidth}{0.5pt}
\renewcommand{\footrulewidth}{0.5pt}
\addtolength{\headheight}{2pt} % make space for the rule
\fancypagestyle{plain}{%
   \fancyhead{} % get rid of headers
   \renewcommand{\headrulewidth}{0pt} % and the line
   \renewcommand{\footrulewidth}{0pt}
}
\fancypagestyle{blank}{%
   \fancyhf{} % get rid of headers and footers
   \renewcommand{\headrulewidth}{0pt} % and the line
   \renewcommand{\footrulewidth}{0pt}
}
\fancypagestyle{abstract}{%
   \fancyhead{}
   \renewcommand{\headrulewidth}{0pt}
   \renewcommand{\footrulewidth}{0.5pt}
}
\fancypagestyle{document}{%
	\fancyhf{} \fancyhead[LE]{\bfseries\nouppercase{\leftmark}}
	\fancyhead[RO]{\bfseries\nouppercase{\rightmark}}
	\fancyfoot[LE,RO]{\bfseries\thepage}
	\renewcommand{\headrulewidth}{0.5pt}
	\renewcommand{\footrulewidth}{0.5pt}
	\addtolength{\headheight}{2pt} % make space for the rule
}
\setcounter{secnumdepth} {5}
\setcounter{tocdepth} {5}
\renewcommand{\thesubsubsection}{\thesubsection.\Alph{subsubsection}}


%for submmited but yet to be accepted publications
\newcommand{\noop}[1]{}