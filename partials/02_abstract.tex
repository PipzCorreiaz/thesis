\pdfbookmark{Abstract}{Abstract}
\begin{abstract}

The computational complexity of some card games attract the interest of Artificial Intelligence (AI) researchers.
Their main challenge is to deal with hidden information, nonetheless recent approaches start to overcome this problem, such as Monte-Carlo Methods.
On the other hand, the strong social component every multi-player game presents can also be included in an artificial player through an embodied agent that interacts with other players.
Therefore, this thesis proposes the development of a social \emph{Sueca} player that is able of both playing the game and communicate with human players, enhancing their game experience.
This agent includes an AI module able of deciding which card to play, based on Perfect Information Monte-Carlo (PIMC) algorithm.
Furthermore, in order to be socially present during the game, this agent also contains a decision maker module able of evaluating the game state and producing adequate verbal or non-verbal behaviours.
Finally, user studies revealed significant comparisons to human players that encourage future development of this work.


\end{abstract}

\begin{keywords}
Artificial Intelligence, Trick-taking Card Game, Hidden Information, Interactive Companions, Socially Intelligent Behaviour
\end{keywords}
\clearpage
\thispagestyle{empty}
\cleardoublepage
