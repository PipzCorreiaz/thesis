\pdfbookmark{Resumo}{Resumo}
\begin{resumo}

A complexidade computacional de alguns jogos de cartas atrai o interesse de investigadores na área da Inteligência Artificial.
Apesar do maior desafio ser a informação escondida, já existem algumas abordagens capazes the ultrapassar este problema, tais como metodologias baseadas em Monte-Carlo.
Por outro lado, a componente social que os jogos com multi-jogadores apresentam é bastante forte e pode, no entanto, ser incluída num jogador artificial através de robôs que interagem com os outros jogadores.
Deste modo, esta tese propõe o desenvolvimento de um jogador social de Sueca que é capaz de jogar o jogo enquanto comunica com jogadores humanos, melhorando a experiência do jogo.
Este agente incluiu um módulo de IA capaz de decidir que carta jogar, com base no algoritmo \emph{Perfect Information Monte-Carlo}.
Para além disso, de maneira a estar socialmente presente durante o jogo, este agente também contém um módulo de decisão capaz de avaliar o estado do jogo e de produzir comportamentos verbais e não verbais adequados.
Por fim, estudos com utilizadores revelaram comparações significativas com jogadores humanos, incentivando o futuro desenvolvimento deste trabalho.
\end{resumo}

\begin{palavraschave}
Inteligência Artificial, Jogos de cartas, Informação escondida, Companheiro Interactivos, Comportamentos socialmente inteligentes
\end{palavraschave}

\clearpage
\thispagestyle{empty}
\cleardoublepage