\fancychapter{Introduction}
\label{sec:introduction}

The world population is ageing dramatically and as such, the society needs to change and embrace this problem in many different ways. The elderly have specific needs, physical and cognitive, that are often not considered in the way we, as a society, organise our lives.
Some of these concerns are recently being solved with the help of technology and may range from computer programs to intelligent robots.
For instance, elderly with mobility disabilities may be guided by a robot while walking through the house \cite{Pollack2002}.
Another example is software for stimulating and training memory problems of Aphasia, a language disorder caused by brain damage \cite{Pompili2011}.

However, existing technology with elderly purposes is commonly focused on health care.
When dealing with aged people with no serious health problems and that are still capable of doing their regular daily tasks, there is still a need to occupy their free time with pleasuring activities.
Finding appropriate tasks supported by technology may include the training of their cognitive functions or just accompany them.
In order to join these two purposes of accompany and train their reasoning, an artificial game player could be a suitable solution.
Several embodied agents for game playing exist, such as the iCat chess tutor \cite{Affective2007} and the \ac{emys} Risk player \cite{Pereira}.%, nevertheless these scenarios were not applied to aged people.
These examples have inspired the idea of exploring a card game scenario, that can exemplify an activity that aged people enjoy doing.
%Several embodied agents for game playing exist, such as the iCat chess tutor \cite{Affective2007} and the \ac{emys} Risk player \cite{Pereira}, nevertheless these scenarios were not applied to aged people.

Overall, a card game is an entertainment activity that aged people are used to do and, at the same time, might help them training their cognitive functions.
As a result, considering some of existing card games are still unsolved challenges for \ac{ai}, the goal of this project is to integrate a social robot with aged humans in a card game scenario.
This project is a great opportunity to relate all the concerns mentioned above.
%On one hand, the proposed agent should play the game correctly, which means, after analysing the given hand, dealt at the beginning of the game, it should make good choices about what cards should be played.
%On the other hand, this embodied agent should act accordingly to the environment of this scenario and every player should be engaged in the game with its interaction and behaviours.
A game-playing companion for elder people must, at the same time, (i) be able to play competently the card game; and (ii) interact socially. The first skill requires such agent to include an AI module that is able to reason strategically about the game. The second skill requires an emotional/social module that enables the agent to behave in a manner that is socially believable.


Computer programs that play games have been an interesting challenge for \ac{ai}.
From board to card games, or even role-playing games, the goal is to create rational agents capable of evaluating the game and achieving the best outcome.
Deep Blue, Chinook and Watson are good examples that have raised the bar for developing this kind of agents.
Deep Blue is a remarkable chess player and has defeated the human world champion in 1997 \cite{Campbell2002}.
Schaeffer et al. have solved Checkers with Chinook program and proved the game leads to a draw with two optimal players \cite{Schaeffer1996}.
Lastly, Watson is the \ac{qa} system that has beat the two highest ranked Jeopardy players in 2011 \cite{Ferrucci2010}.
%\ac{qa} systems are extremely impressive due to the scope of \ac{ai} they include (i.e. natural language processing, machine learning, knowledge representation, automated reasoning, and information retrieval).
All these agents are good baselines to improve \ac{ai} in games.


Besides building programs that try to think rationally or humanly, \ac{ai} has also another branch that aims to act humanly \cite{Russell2009}.
This concern arises from the inclusion of robots in humans' life and influences the way they interact and communicate with people.
Consequently, robots have to behave properly in those environments, considering they are surrounded by humans.
The \ac{hri} field explores these concerns of integrating robots with humans in a social environment.
Since this field descends from \ac{hci}, it also inherits the user centred development that establishes the prior need of doing user studies.


%Consequently, technology that covers this point has concerns related with the feeling of objectification, loss of responsibilities and privacy \cite{Should2010}.



Lastly, the chosen card game is \emph{Sueca}, a well known Portuguese game among the ageing population.
It is a four-player game with two teams and involves an opponent and a companion role for the agent.
Regarding \ac{hri} concerns, these two roles together have not been studied yet in an artificial embodied game player.



\section{Goals}
\label{sec:goals}

The main goals of this project are:
\begin{itemize}
\item To develop a robotic agent capable of playing competently the \emph{Sueca} card game;
\item To include social behaviours on an embodied agent in order to act according to the game state;
\item To evaluate the correctness and advantages of the proposed system.
\end{itemize}

\section*{\centering*}

\todo[inline]{Review the outline of the document}
The next section presents some background research that helps the reader understand the problems further mentioned (Section~\ref{sec:background}).
The report proceeds with the state-of-art of playing card-games and human-robot-interaction with the elderly (Section~\ref{sec:related-work}).
Additionally, it reveals a pilot user study (Section~\ref{chapter:user-studies}).
Finally, it presents the architecture, its evaluation methodology and the final conclusions.
