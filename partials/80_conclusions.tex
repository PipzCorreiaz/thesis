\fancychapter{Conclusions}
\label{chapter:conclusion}

In order to address this thesis problem, presented in Section~\ref{sec:problem}, we started by reviewing state of the art approaches to both artificial intelligences for card games and \acp{hri} in games.
Theses research conducted the implementation of the \ac{pimc} algorithm on our artificial \emph{Sueca} player.
Additionally, this intelligent player was included as a module of an architecture for a social \emph{Sueca} player.
This social entity was able of playing the card game with human players while interacting with them according to game state.
Furthermore, user studies evidenced some very positive aspects of the developed player.

\section{Future work}
The future work for enhancing the artificial \emph{Sueca} player starts by testing the results of other reviewed algorithms.
In addition, modelling opponents would also be a great improvement through machine learning algorithms.
This idea combines with Monte-Carlo Methods, since it would decrease the numerous sampling requirements.
Furthermore, considering the gap of social robots on elderly population, as reviewed in this thesis, it would be interesting to target this \emph{Sueca} player for older adults.