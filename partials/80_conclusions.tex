\fancychapter{Conclusions}
\label{chapter:conclusion}

This thesis addressed three main contributions aligned with the problems presented in Section~\ref{sec:problem}.
First of all, the implementation of the \ac{pimc} algorithm on an artificial \emph{Sueca} player and later analysis on different parametrizations of this algorithm.
Additionally, this intelligent player was included as a module of an architecture for a social \emph{Sueca} player.
This social entity was able of playing the card game with human players while interacting with them according to game state.
Finally, we conducted user studies to compare trust and social presence between human partners and \ac{emys}, and also a affect evolution after the game.

\section{Future work}
The future work for enhancing the artificial \emph{Sueca} player starts by testing the results of other reviewed algorithms.
In addition, modelling opponents would also be a great improvement through machine learning techniques.
This idea combines with Monte-Carlo Methods, since it would decrease the numerous sampling requirements.
Furthermore, considering the gap of social robots on elderly population, as reviewed in this thesis, it would be interesting to target this \emph{Sueca} player for older adults.