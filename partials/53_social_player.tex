\section{A social player}
\label{chapter:social-player}

Creating a social player in a card game scenario has many different challenges.
This player, embodied in an expressive robot, has to interact in a proper way according to the game situation, as similar as possible to the interactions of human players.
Additionally, a \emph{Sueca} player has two distinct roles during a game: being partner of his team player and opponent of the other two players.
With these main concepts in mind, a character of this nature can be developed by specifying its behaviours.
In other words, considering the connection between the world and the robot is established by \emph{Skene}, this means creating a list of utterances that might contain text to speak, animations, gaze and glance instructions.
The current section describes the structure of behaviours included in the developed \emph{Sueca} player.


\subsection{\emph{Sueca} behaviours}
The analysis of the card game players on user centred studies revealed key aspects of the interaction during a \emph{Sueca} game.
First of all, there are specific game situations that may cause verbal or non-verbal behaviours.
As a result, these game situations guided the list of events that trigger


\subsection{Human-like behaviours}
frequency
emotions

\subsection{Enhancing the game interface with behaviours}
nextplayer
receiving cards
trick end