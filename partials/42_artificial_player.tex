\section{An intelligent player}%The artificial player}
\label{sec:artificial_player}

This section will describe the most relevant implementation details of the artificial player.
After thoroughly analysing state-of-the-art techniques to solve imperfect information games, and considering \emph{Sueca} is, at this moment, computationally unsolved, the chosen approach was \ac{pimc}.
To implement this search technique, there are three key concepts or algorithms that require a full understanding: the Information Set, the PICM Search and the MinMax Algorithm.

\subsection*{Information Set}

An information set represents all the visible information during a game, and also inferred information based on certain events.
The player must keep an instance of the information set per game and update it when necessary.
It stores the known hand of the player and a deck with all the cards whose owner is unknown.
As result, each time another player plays a card, it should be removed from that deck.

The purpose of managing unplayed cards is to sample possible card distributions for the other three players with their real conditions.
These sampled distributions will be used during the \ac{pimc} search and the closer they are to the real world, the better the search returning value will be.
Additionally, the information set keeps track of suits per player and, when a player does not follow the leadsuit of a trick, it removes that suit from the player possible suits.
By possessing this information, sampling possible distributions gets closer to the real world, however it increases the complexity of the sampling process.
The method builds a \ac{csp} where:
\begin{itemize}
\item variables are the unplayed cards;
\item each domain is the set of players that still have that suit;
\item and the constraints are the number of times a player can be assigned to a card.
\end{itemize}


\subsection*{\ac{pimc} Search}

The following pseudo-code of the \ac{pimc} search algorithm guided the implementation.

\begin{algorithm}
	\caption{PIMC search algorithm}
	\begin{algorithmic}[1]
		\Procedure{PIMC}{InfoSet $I$, int $N$}
			\ForAll {$m \in$ Moves($I$)}
				\State $val[m]$ = 0
			\EndFor
			\ForAll {$i \in \{ 1..N\}$}
				\State $x$ = Sample($I$)
				\ForAll {$m \in$ Moves($I$)}
					\State $val[m]$ += PerfInfoValue($x$, $m$)
				\EndFor
			\EndFor
			\State \textbf{return} $\underset{m}{argmax}\{ val[m] \}$
		\EndProcedure
	\end{algorithmic}
\end{algorithm}

To recapitulate the main point of this algorithm, considering it can choose up to \#Moves($I$), it samples $N$ possible card distributions for the other three players and calculates the reward of playing each possible move for the $N$ sampled worlds.
The returned move is the one that gave more accumulated reward.

The number of iterations this algorithm perform is imposed by the $N$ parameter.
Another version of the algorithm, instead of limiting the number of iterations, specifies the execution time of the main loop.

\subsection*{MinMax Algorithm}

As mentioned above, \ac{pimc} has to calculate the reward of playing a card, for each sampled world.
Since a sampled distribution assigns the remaining cards to the players, every game can be handled as a perfect information game.
Therefore, to compute a perfect information game, considering each player or team intends to win, the MinMax algorithm was used.

MinMax is a popular algorithm for calculating optimal decisions in multiplayer games.
Each node corresponds to a possible move by a player and their successors correspond to the possible moves of next player.
In the \emph{Sueca} game tree, each group of four levels represents a trick.
Since the utility value can only be determined in terminal nodes, these back-propagate their best or worst child utilities, if they are max or min nodes, respectively.



\subsection{Drawbacks}

Bla bla bla


\subsection{Enhancements}

Bla bla bla
