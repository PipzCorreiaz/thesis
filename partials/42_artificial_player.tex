\section{An intelligent player}%The artificial player}
\label{sec:artificial_player}

This section will describe the most relevant implementation details of the artificial player.
After thoroughly analysing state-of-the-art techniques to solve imperfect information games, and considering \emph{Sueca} is, at this moment, computationally unsolved, the chosen approach was \ac{pimc}.
To implement this search technique, there are three key concepts or algorithms that require a full understanding: the Information Set, the PICM Search and the MinMax Algorithm.

\subsection*{Information Set}

An information set represents all the visible information during a game, and also inferred information based on certain events.
The player must keep an instance of the information set per game and update it when necessary.
It stores the known hand of the player and a deck with all the cards whose owner is unknown.
As result, each time another player plays a card, it should be removed from that deck.

The purpose of managing unplayed cards is to sample possible card distributions for the other three players with their real conditions.
These sampled distributions will be used during the \ac{pimc} search and the closer they are to the real world, the better the search returning value will be.
Additionally, the information set keeps track of suits per player and, when a player does not follow the leadsuit of a trick, it removes that suit from the player possible suits.
By possessing this information, sampling the distributions gets closer to the world, however it increased the complexity of the sampling process.
The method builds a \ac{csp} where:
\begin{itemize}
\item variables are the unplayed cards;
\item each domain is the set of players that still have that suit;
\item and the constraints are the number of times a player can be assigned to a card.
\end{itemize}


\subsection*{\ac{pimc} Search}

The following pseudo-code of the \ac{pimc} search algorithm guided the implementation.
To recapitulate the main point of this algorithm, considering it can choose up to \#Moves($I$), it samples $N$ possible card distributions for the other three players and calculates the reward of playing each possible move for the $N$ sampled worlds. The returned move is the one that gave more accumulated reward.

\begin{algorithm}
	\caption{PIMC search algorithm}
	\begin{algorithmic}[1]
		\Procedure{PIMC}{InfoSet $I$, int $N$}
			\ForAll {$m \in$ Moves($I$)}
				\State $val[m]$ = 0
			\EndFor
			\ForAll {$i \in \{ 1..N\}$}
				\State $x$ = Sample($I$)
				\ForAll {$m \in$ Moves($I$)}
					\State $val[m]$ += PerfInfoValue($x$, $m$)
				\EndFor
			\EndFor
			\State \textbf{return} $\underset{m}{argmax}\{ val[m] \}$
		\EndProcedure
	\end{algorithmic}
\end{algorithm}


\subsection*{MinMax Algorithm}

Bla bla bla


\subsection{Drawbacks}

Bla bla bla


\subsection{Enhancements}

Bla bla bla
