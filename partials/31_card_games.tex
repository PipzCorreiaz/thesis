\section{AI in games}

%%% jogos tem sido um grande desafio
\ac{ai} has been solving many games over the years, however, the definition of ``games'' usually refers to zero-sum and perfect information games.
These kind of games are commonly solved by creating a tree representing all possible states and searching for the optimal, or a nearly optimal, solution.
The greatest achievements related to perfect information games are generally based on finding good heuristics to refine the search and also good prunings to reduce the search space.
Deep Blue can exemplify this idea \cite{Campbell2002}, it uses an iterative-deepening alpha-beta search and the key of its success is mostly the null move heuristic and the futility pruning.
Another example is Chinook, also a perfect information game that was solved using alpha-beta search \cite{Schaeffer1996}.

%%% perfect e imperfect nao se resolvem da mesma maneira
Nevertheless, \emph{Sueca} is considered an imperfect information game, as described in Section~\ref{sec:sueca}, and this class of games is usually solved by one of three different approaches \cite{Cowling2012}.
The first one, and the most popular, is based on Monte-Carlo Methods.
Then, another possible approach is trying to compute a Nash equilibrium strategy or an approximation thereof.
Lastly, belief distributions involving game state inferences and opponent models can also be used.
The first two mentioned approaches are mutually exclusive, while the last one can be used as a supplement.
The following two subsections detail how Monte-Carlo Methods and Belief Distributions can be applied in hidden information games.
The second pointed approach will not be addressed due to the imposed limitations of our domain, considering, for instance, that the maximum known number of states for computing a Nash equilibrium is $10^{12}$ \cite{Zinkevich}, which is much lower than the number of possible states in a \emph{Sueca} game.




\subsection{Monte-Carlo Methods}

%%% monte carlo e cada vez mais popular
The popularity and acceptance of Monte-Carlo based methods have increased since its success on Bridge.
\ac{gib}\footnote{http://www.gibware.com/} was the first computer bridge champion using Monte-Carlo Methods, and subsequently, another two successful domains were Skat\footnote{https://skatgame.net/} and Computer Go \cite{Gelly2011}.
Since some of these domains remain a challenge for traditional \ac{ai} techniques, this method seems to be very promising.


%%% diferenca entre MCTS e PIMC
In order to solve a hidden information game, the first challenge is to deal with information sets.
The most used approach to solve it is determinization, which samples choice nodes instead of considering all of them in an unique set.
The combination of this approach to Monte-Carlo Methods is known as \ac{pimc}.
For instance, in a card game scenario, each iteration of \ac{pimc} samples the cards distributions for all players and the simulation process of the game behaves as a perfect information game.
In other words, during the simulation each player makes decisions as if his opponents' cards are visible.
The first successful implementation of this technique was \ac{gib} \cite{Ginsberg2001}.


%%% Basin estudou os prblomeas do PIMC
In 1998, \citet*{Frank1998} produced an analysis on \ac{pimc}'s limitations.
They identified two distinct problems: \emph{strategy fusion} and \emph{non-locality}.
Due to the repeated minimaxing architecture that \ac{pimc} has and its evaluation of possible distributions with the best strategy, applying this knowledge, when information is missing, might produce suboptimal decisions.
This is called the \emph{strategy fusion}.
For instance, when having a move with a guaranteed reward and another move with a possible reward of the same value although depending on the current world, \ac{pimc} equally considers both moves.


The second problem, \emph{non-locality}, results from the propagation of values.
The value of a game tree node only considers its children' values, however, in an imperfect information game, some guesses might be done using values of the non-local subtree.
For instance, considering 2 different worlds, the player 1 can guarantee a winning trick in the world 1 by making a certain move, and if in that state, he makes another move instead, player 2 might assume they are in world 2.
\ac{pimc} cannot make such an inference.


%%% O Long estudou com que propriedades o PICM funciona
Despite the satisfying outcomes of \ac{pimc}, there were still difficulties in understanding the strong results of this algorithm.
As such, Long et al. have analysed the previously mentioned problems of \ac{pimc} search, and they have shown how three different properties of a game can influence the success of \ac{pimc} \cite{Long2010}.
The first property is \emph{leaf correlation}, which refers to how likely it is to affect a player's payoff in the neighbourhood of a leaf.
When the probability of all siblings having the same payoff values is higher, the correlation value increases.
Secondly, \emph{bias} indicates the chance of a player being preferred over another.
Finally, the last game characteristic that has been pointed is \emph{disambiguation factor}, that denotes how rapidly the hidden information is revealed.


These properties have been tested in a set of experiments in both \ac{pimc} and a random player against an optimal Nash equilibrium player.
Results shown the performance of \ac{pimc} increases as the correlation value is higher, bias does not considerably affect its success, and, finally, disambiguation has the greatest impact on the results of the algorithm.
When this last value is higher, it means the game turns more quickly into a perfect information game.
Additionally, the authors demonstrate these properties on real game examples, such as Skat and Kuhn poker.
Skat indicates a considerably good performance of \ac{pimc}, due to its values of \emph{leaf correlation}, \emph{bias}, and \emph{disambiguation factor}.
Since Skat presents strong similarities to \emph{Sueca}, it is expected that \ac{pimc} also has a good performance when applied to \emph{Sueca}.


%%%  cowling propose ISMCTS
Cowling et al. have also investigated the application of \ac{mcts}\footnote{\ac{mcts} algorithm builds a search tree according to the results of previous iterations by sampling unknown informations.} to hidden information games \cite{Cowling2012}.
Their research supports a new descendant family of algorithms, \ac{ismcts}.
\ac{ismcts} works with information sets, instead of game states and uses determinization to sample the game, however producing a single tree.
The main advantages are the computational budget efficiency and the fact of suffering less from \emph{strategy fusion} than \ac{pimc}.
The authors also presented some experiments in three different games, including a card game.
Their results on the card game Dou Di Zhu were very similar to \ac{uct} and did not introduce any improvement to the playing strength.
The authors explained these results with the high branching factor this domain produces, which has discouraged the usage of this technique on the domain of \emph{Sueca}, since in the information set tree, the initial branching factor would also be high ($10^{8}$, $\Comb{40}{10}$).


%%% IIMC
Recently, Furtak \& Buro \cite{Furtak} presented a new search algorithm called \ac{iimc} that can be suitably applied to hidden information games and reduces the \emph{strategy fusion} problem.
During the simulation phase, each player's move is chosen inside a player's module and the game behaves as an imperfect information due to this encapsulation.
Additionally, the players' modules allow the differentiation of players using different strategies.
The authors revealed the great potential of this approach when applied to trick-based card games, considering it has been successfully tested in the Skat scenario.


\begin{table}[h]
\caption{Advantages and disadvantages of the mentioned Monte-Carlo algorithms}
\resizebox{\textwidth}{!}{%
\begin{tabular}{|c|l|l|}
\hline
\textbf{Algorithm} & \multicolumn{1}{c|}{\textbf{Advantages}} & \multicolumn{1}{c|}{\textbf{Disadvantages}}\\
\hline
\ac{pimc}      & \begin{tabular}[c]{@{}l@{}}Offline Computation\\ Easy to parallel\end{tabular}                                             & \begin{tabular}[c]{@{}l@{}}Strategy fusion\\ Non-locality\end{tabular}                                        \\ \hline
\ac{ismcts}    & \begin{tabular}[c]{@{}l@{}}Offline Computation\\ Easy to parallel\\ Computational budget\end{tabular}                      & \begin{tabular}[c]{@{}l@{}}Strategy fusion (less than \ac{pimc})\\ Non-locality\\ Complexity\end{tabular}                       \\ \hline
\ac{iimc}      & \begin{tabular}[c]{@{}l@{}}Offline Computation\\ Easy to parallel\\ Allow a different player model per player\end{tabular} & \begin{tabular}[c]{@{}l@{}}Strategy fusion (less than \ac{pimc})\\ Non-locality\\ Complexity\end{tabular} \\ \hline
\end{tabular}
}
\label{tab:algorithms}
\end{table}


%Due to the similarity between \emph{Sueca} and Skat, it is predictable that Monte-Carlo methods might produce good results in our domain.
%However, assuming that different \ac{mcts} variations produce the exact same behaviour, in different domains, might be a mistake, considering they have specific characteristics. % esta errado

The advantages and disadvantages of the mentioned \ac{mcts} variations for imperfect informations games are clearly summarised in Table~\ref{tab:algorithms}.
Both of three techniques are easy to parallel and allow an offline computation. However, \ac{ismcts} uses the computational budget more efficiently than the other two techniques, and \ac{iimc} allows different player models per player.
Disadvantages show all the three techniques have the \emph{non-locality} and \emph{strategy fusion} problem, although \emph{strategy fusion} is lower in \ac{ismcts} and \ac{iimc}.
%Another relevant disadvantage is the computational burden that \ac{ismcts} and \ac{iimc} add, when compared to \ac{pimc}.
%Despite having a better computational budget, applying \ac{ismcts} to \emph{Sueca} should produce a similar behaviour to Dou Di Zhu, due to the high branching factor in the information set tree.
%The usage of different player models can be helpful to turn the game to an imperfect information game and, consequently, reduce the \emph{strategy fusion} problem.

Overall, the chosen approach for the \emph{Sueca} domain is \ac{pimc}, since it will be used at runtime and this algorithm provides the lightest computational burden to do that.



\subsection{Game State Inference \& Opponent Modelling}


While discussing imperfect information games, belief distributions, game state inference and opponent modelling are other relevant subjects to consider.
Predicting some of the opponents' cards or other clues would be beneficial to select better actions at each state of the game. Additionally, inferring hidden information, while using a Monte-Carlo based method, can also decrease the \emph{non-locality} problem \cite{Cowling2012}.
%This problem is known as finding the distribution $P$(world\textbar move) for a move played in an hypothetical world.


Buro in 2009 \cite{Buro} presented his work on state evaluation and inference that has been included in his Skat player.
His approach combines two techniques, one for evaluating the bidding and another for selecting hypothetical worlds during the game play.
The former technique uses logistic regression to evaluate the winning probability of each hand and it has 22 million Skat games as data base.
This winning probability determines the strength of a hand and can, therefore, be used on the bidding.


The second technique is mainly based on two heuristics.
Fastest-cut-first search heuristic evaluates each move according to its beta-cutoff value and minimises the expected number of visited nodes.
Additionally, in order to reduce the tree exploration, another heuristic groups cards by their strength value and considers, for example, 7\ding{168} and 8\ding{168} the same move, when holding both cards in a player's hand.
The author compares his work to other similar ones and concludes the strength of his techniques lies in two central points.
First, determining the $P$(world\textbar move) on offline data, instead of doing it in runtime.
Sencond, his formulation is generalised in a way that it is possible to perform it on high-level features.
Since the main difference between \emph{Sueca} and Skat is that the first one does not have the bidding phase, Buro's first technique would not be appropriate for the \emph{Sueca} game.
However, the search enhancements could be suitably applied, considering the game trees are identical.


Usually, opponent modelling uses optimal strategies to predict the other players' actions and these models tend to be overly defensive.
Consequently, Long \& Buro in 2011 \cite{Long2009} suggested a post-processing analysis that is able to infer opponent's qualities based on their decisions in a certain environment.
The main idea is to classify each opponent with a mistake rate and use that value to be more or less defensive.
This approach, called \ac{pipma}, computes a procedure after each game episode (in a trick-taking card game, it would be after each trick) to incrementally update the mistake rate of each opponent.
The authors made some experiments in a Skat player with very good results, where they used the mistake rate to adjust the bidding behaviour during the game.
%Likewise, the search improvements reduced about 40\% of the search space.
Despite the fact that \emph{Sueca} does not have the bidding phase, classifying opponents with a mistake rate can useful to other purposes.
As a result, it would be interesting to model the opponents in a similar way in the domain of \emph{Sueca}, in order to make better decisions or even for the embodied agent to produce adequate behaviours.

Another highly suitable card game to make opponent models is Poker, since predicting the players' moves can naturally affect the outcome of this game.
In order to predict players' cards and their future actions, Posen et al. in 2010 \cite{Ponsen2008} have investigated this subject.
They proposed an opponent model that starts with a prior distribution and changes over time with a differentiating function.
The prior distribution allows it to make reasonable inferences while having insufficient information.
In addition, the relational probability tree algorithm TILDE builds a decision tree with the stored samples of a player.
This decision tree represents the differentiating function that will adapt the initial prior distribution.
Besides this opponent model, the authors explain how to integrate this function with \ac{mcts}.
Instead of sampling the cards randomly, \ac{mcts} uses card predictions and, therefore, the algorithm does not need a numerous amount of iterations to reach a uniform card distribution.
Furthermore, the probabilities of action predictions are used in the selection phase of the \ac{mcts}, according to the state of the game and the sampled cards.
Since \ac{mcts} can be used in the \emph{Sueca} domain, a similar opponent model can also improve the capabilities of this algorithm, as shown in Poker.




\begin{table}[h]
\caption[Reviewed techniques and their goals]{Techniques signed with \ding{53}, \ding{51} and $\sim$ symbols are, respectively, not suitable, suitable and conditionally suitable to the \emph{Sueca} domain.}
\resizebox{\textwidth}{!}{%
\begin{tabular}{|C{0.1\textwidth}|l|l|C{0.12\textwidth}|}
\hline
\textbf{Tested domain} & \multicolumn{1}{c|}{\textbf{Technique}}     & \multicolumn{1}{c|}{\textbf{Goal}} & \textbf{Suitable to \emph{Sueca}} \\ \hline
\multirow{4}{*}{Skat}  & Determine the winning probability of a hand & Improve the bidding                & \ding{53}                         \\ \cline{2-4}
                       & Fastes-cut-first heuristic                  & Order moves                        & \ding{51}                         \\ \cline{2-4}
                       & Considering similar states equally          & Reduce tree exploration            & \ding{51}                         \\ \cline{2-4}
                       & Calculate the mistake rate of each player   & Improve the bidding                & $\sim$ \\ \hline
Poker                  & Opponent model                              & Improve \ac{mcts} policies      & \ding{51}                         \\ \hline
\end{tabular}
}
\label{tab:belief}
\end{table}



Table~\ref{tab:belief} summarises what techniques have been reviewed, their purposes, and, finally, if they can be applied to the \emph{Sueca} domain.
The technique of determining the winning probability cannot be used for the exact same purpose, since our domain does not include a bidding phase.
The next two search enhancements can naturally be used due to the similarities between Skat and our domain game trees.
The mistake rate was signed as conditionally suitable because it can also be used, although with a different purpose.
Thinking in the embodied agent of our work, it can assign a mistake rate variable to each player and produce appropriate behaviours according to their values.
A similar approach might be thought to use the winning probability, however, opponents' hands are not visible and the agent should not reveal its own information.
The last technique also can be an addition to the \ac{mcts} base policies.


